\documentclass[aspectratio=169]{beamer}

% --- Paquetes y Configuración ---
\usepackage[utf8]{inputenc}
\usepackage[spanish]{babel}
\usepackage{graphicx}
\usepackage{tikz}
\usepackage[backend=biber, style=ieee, bibstyle=numeric, citestyle=numeric, sorting=none]{biblatex}
\addbibresource{references.bib}

% --- Estética del Beamer ---
\setbeamertemplate{sidebar right}{}
\setbeamertemplate{footline}{%
  \hfill\usebeamertemplate***{navigation symbols}
  \hspace{1cm}\insertframenumber{}/\inserttotalframenumber}
\setbeamertemplate{bibliography item}{\insertbiblabel}

% --- Datos de Portada ---
\title{Sistema de Gestión de Solicitudes Mesa de Ayuda "Helpdesk".}
\subtitle{Tesina para obtener el grado de Ingeniería en Tecnologías de la Información.}
\author{Juan Daniel Guerrero Guerrero}
\institute{Universidad Politécnica de Victoria}
\date{8 de Diciembre de 2025}

\begin{document}

% --- Diapositiva 1: Portada ---
{
\usebackgroundtemplate{\tikz\node {\includegraphics[height=\paperheight,width=\paperwidth]{Diapositiva1}};}
\begin{frame}[plain]
    \titlepage
\end{frame}
}

% --- Índice (Opcional, pero útil) ---
\begin{frame}{Contenido}
    \tableofcontents
\end{frame}

% --- Diapositiva 2: Definición del Problema (Fondo especial) ---
{
\usebackgroundtemplate{\tikz\node {\includegraphics[height=\paperheight,width=\paperwidth]{Diapositiva2}};}
\section{Introducción}
\begin{frame}{Situación Actual en la DIT}
    \begin{itemize}
        \item Actualmente, el proceso de atención en la DIT carece de un sistema organizado.
        \item Se utilizan canales informales (WhatsApp, verbal), lo que causa pérdida de información.
        \item Existe duplicidad de tareas y falta de indicadores para la toma de decisiones.
    \end{itemize}
\end{frame}
}

% --- Diapositiva 3: Objetivos ---
\begin{frame}{Objetivos}
    \textbf{Objetivo General}
    \begin{itemize}
        \item[] Desarrollar un sistema web que permita gestionar de manera eficiente, centralizada y trazable las solicitudes.
    \end{itemize}

    \vspace{0.3cm}

    \textbf{Objetivos Específicos}
    \begin{itemize}
        \item Implementar módulos de autenticación y asignación.
        \item Generar reportes y visualización de indicadores (KPIs).
    \end{itemize}
\end{frame}

\section{Marco Teórico}
\begin{frame}{Tecnologías y Fundamentos}
    Para el desarrollo del sistema se seleccionaron tecnologías robustas y de código abierto:
    \vspace{0.3cm}
    \begin{itemize}
        \item \textbf{Lenguaje:} PHP 8.x con compilación JIT.
        \item \textbf{Framework:} CakePHP (Arquitectura MVC).
        \item \textbf{Base de Datos:} MySQL (Relacional, 3NF).
        \item \textbf{Frontend:} Bootstrap y HTML5.
        \item \textbf{Seguridad:} Encriptación Argon2id y protección CSRF.
    \end{itemize}
\end{frame}


% ==========================================
% SECCIÓN 2: Sistema Propuesto
% ==========================================
\section{Sistema Propuesto}

% --- Diapositiva 4: Arquitectura MVC (Movida aquí por lógica) ---
\begin{frame}{Arquitectura del Sistema (MVC)}
    Se utiliza una arquitectura Cliente-Servidor sobre el framework CakePHP.
    \begin{itemize}
        \item \textbf{Modelo:} Gestiona la lógica de datos y conexión a MySQL.
        \item \textbf{Vista:} Plantillas que generan la interfaz de usuario.
        \item \textbf{Controlador:} Intermediario que procesa las peticiones HTTP.
    \end{itemize}

    \vspace{0.3cm}
    \begin{center}
        \includegraphics[width=0.8\textwidth]{diagrama_mvc_uml.png}
    \end{center}
\end{frame}

% --- Nueva Diapositiva: Base de Datos ---
\begin{frame}{Diseño de Base de Datos}
    Se implementó un esquema relacional normalizado (3NF) en MySQL.
    \vspace{0.3cm}
    \begin{columns}
        \column{0.5\textwidth}
        \textbf{Entidades Principales:}
        \begin{itemize}
            \item \texttt{TrnMsolicitudes}: Ticket maestro.
            \item \texttt{TrnDsolicitudes}: Bitácora de seguimiento (Detalle).
            \item \texttt{Users} y \texttt{Roles}: Control de acceso (RBAC).
        \end{itemize}
        
        \column{0.5\textwidth}
        \begin{center}
            % Asegúrate de tener una imagen del MER o diagrama relacional
            \includegraphics[width=0.9\textwidth]{Diagrama_ER.png} 
        \end{center}
    \end{columns}
    \vspace{0.2cm}
    \footnotesize{La relación maestro-detalle permite la trazabilidad completa del historial del ticket.}
\end{frame}
% --- Nueva Diapositiva: Base de Datos ---

% --- Diapositiva 5: Actores ---
\begin{frame}{Actores y Funcionalidades: Flujo Operativo}
    El sistema se basa en la interacción diaria entre quien tiene el problema y quien lo resuelve.

    \vspace{0.2cm}
    \textbf{Actores Operativos:}
    \begin{itemize}
        \item \textbf{Solicitante:} Reporta incidencias y da seguimiento.
        \item \textbf{Agente Técnico:} Diagnostica, comenta y resuelve.
    \end{itemize}

    \vspace{0.2cm}
    \begin{center}
        % Ajustamos al 85% del ancho del texto para que se vea grande
        % Asegúrate de que la imagen esté en la carpeta del proyecto
        \includegraphics[width=0.3\textwidth]{diagrama_caso_de_uso_1.png}
    \end{center}
\end{frame}

\begin{frame}{Actores y Funcionalidades: Nivel de Gestión}
    Por encima del flujo operativo, existen roles de control y configuración para garantizar la calidad del servicio.

    \vspace{0.2cm}
    \textbf{Actores de Gestión:}
    \begin{itemize}
        \item \textbf{Jefe de Área:} Realiza el triaje (cambio de área) y valida soluciones.
        \item \textbf{Administrador:} Configura catálogos y usuarios globales.
    \end{itemize}

    \vspace{0.2cm}
    \begin{center}
        % Mismo ajuste de tamaño
        \includegraphics[width=0.28\textwidth]{diagrama_caso_de_uso_2.png}
    \end{center}
\end{frame}


% --- Diapositiva 6: Seguridad ---
\begin{frame}{Autenticación y Control de Acceso}
    \begin{itemize}
        \item Implementación de autenticación segura y control de acceso basado en roles (RBAC).
        \item Validación de identidad mediante usuario y contraseña encriptada.
    \end{itemize}
    \vspace{0.2cm}
    \begin{center}
        \includegraphics[width=0.5\textwidth]{Login.png}
    \end{center}
\end{frame}

% --- Diapositiva 7: Registro ---
\begin{frame}{Registro de Solicitudes}
    \begin{itemize}
        \item Formulario accesible para usuarios internos y externos.
        \item Captura de datos críticos: Descripción, tipo de solicitud y evidencia.
        \item Validación automática de archivos adjuntos.
    \end{itemize}
    \vspace{0.2cm}
    \begin{center}
        \includegraphics[width=0.5\textwidth]{Formulario.png}
    \end{center}
\end{frame}

% --- Diapositiva 8: Automatización ---
\begin{frame}{Automatización y Notificaciones}
    \begin{itemize}
        \item Envío automático de correos electrónicos al crear un ticket o cambiar su estatus.
        \item Incluye un enlace directo (token) para seguimiento sin necesidad de login.
    \end{itemize}
    \vspace{0.2cm}
    \begin{center}
        \includegraphics[width=0.7\textwidth]{Notificación.png}
    \end{center}
\end{frame}

% ==========================================
% SECCIÓN 4: MÓDULOS Y RESULTADOS
% ==========================================
\section{Implementación y resultados}

% --- Diapositiva 9: Dashboard ---
\begin{frame}{Panel de Control (Dashboard)}
    \begin{itemize}
        \item Visualización de KPIs en tiempo real.
        \item Contadores de tickets asignados, nuevos y pendientes.
        \item Acceso rápido a reportes por área o agente.
    \end{itemize}
    \vspace{0.2cm}
    \begin{center}
        \includegraphics[width=0.5\textwidth]{Dashboard.png}
    \end{center}
\end{frame}

% --- Diapositiva 10: Gestión ---
\begin{frame}{Gestión y Asignación de Tickets}
    \begin{itemize}
        \item Listado general de tickets en ``Recepción''.
        \item \textbf{Flujo:} El jefe de área reasigna al departamento o técnico.
        \item Visualización de estatus: \textit{Nuevo, En Proceso, Finalizado}.
    \end{itemize}
    \vspace{0.2cm}
    \begin{center}
        \includegraphics[width=0.5\textwidth]{Generación_y_visualización.png}
    \end{center}
\end{frame}

% --- Diapositiva 11: Bitácora ---
\begin{frame}{Seguimiento y Bitácora}
    \begin{itemize}
        \item Historial detallado de cada incidencia.
        \item Interacción tipo chat entre técnico y solicitante.
        \item Validación obligatoria del jefe de área para cerrar tickets.
    \end{itemize}
    \vspace{0.2cm}
    \begin{center}
        \includegraphics[width=0.5\textwidth]{vista_seguimiento.png}
    \end{center}
\end{frame}

% --- Diapositiva 12: Admin ---
\begin{frame}{Módulo Administrativo}
    \begin{itemize}
        \item Gestión de catálogos: Usuarios, Áreas, Tipos y Estatus.
        \item Panel exclusivo para rol \textbf{Administrador}.
        \item Configuración global sin tocar código.
    \end{itemize}
    \vspace{0.2cm}
    \begin{center}
        \includegraphics[width=0.5\textwidth]{Vista_admin.png}
    \end{center}
\end{frame}

% --- Diapositiva 13: Técnico ---
\begin{frame}{Vista del Técnico}
    \begin{itemize}
        \item Bandeja personalizada filtrada por ID de empleado.
        \item Acciones rápidas: \textbf{Ver detalles} o \textbf{Cambiar estatus}.
    \end{itemize}
    \vspace{0.2cm}
    \begin{center}
        \includegraphics[width=0.5\textwidth]{gestion_tecnico.png}
    \end{center}
\end{frame}

% --- Nueva Diapositiva: Pruebas ---
\begin{frame}{Pruebas y Validación (Testing)}
    Se realizaron dos niveles de pruebas para garantizar la calidad del software:
    
    \vspace{0.3cm}
    
    \textbf{1. Pruebas Unitarias e Integración (Backend)}
    \begin{itemize}
        \item Validación de reglas de negocio en Modelos y Controladores (PHPUnit).
        \item Verificación de flujo de estados (Ej. \textit{Nuevo} $\rightarrow$ \textit{En Proceso}).
    \end{itemize}
    
    \vspace{0.2cm}
    
    \textbf{2. Pruebas de Aceptación de Usuario (UAT)}
    \begin{itemize}
        \item Validación funcional con personal de la DIT.
        \item \textbf{Resultado:} Se confirmó la trazabilidad y la correcta asignación de tickets entre áreas.
    \end{itemize}
\end{frame}
% ==========================================
% SECCIÓN 5: CIERRE
% ==========================================
\section{Conclusiones y Trabajo Futuro}

% --- Diapositiva 14: Conclusiones ---
\begin{frame}{Conclusiones}
    \begin{itemize}
        \item Se logró centralizar la información y eliminar la pérdida de datos.
        \item Se implementó exitosamente la trazabilidad y auditoría.
        \item El sistema provee métricas reales para evaluar la eficiencia de la DIT.
    \end{itemize}
    \vspace{1cm}
    \begin{center}
        \textbf{Universidad Politécnica de Victoria}
    \end{center}
\end{frame}

% --- Diapositiva 15: Futuro ---
\begin{frame}{Trabajo Futuro}
    \begin{itemize}
        \item \textbf{Módulo de Inventarios:} Vinculación tickets - hardware.
        \item \textbf{Notificaciones Real-Time:} WebSockets para alertas instantáneas.
        \item \textbf{Aplicación Móvil:} App nativa para técnicos de campo.
    \end{itemize}
\end{frame}

% --- Referencias ---
\section*{Referencias}
\begin{frame}[t,noframenumbering,plain,allowframebreaks]{Referencias}
    % \nocite{*} obliga a incluir TODAS las entradas del archivo .bib 
    % aunque no las hayas citado en el texto.
    \nocite{*} 
    
    \AtNextBibliography{\tiny}
    \printbibliography
\end{frame}



% En la ultima diapositiva ponen de fondo la version de la portada
\usebackgroundtemplate{%
%%\tikz\node[opacity=0.3] {\includegraphics[height=\paperheight,width=\paperwidth]{Diapositiva1}};}
\tikz\node {\includegraphics[height=\paperheight,width=\paperwidth]{Diapositiva1}};}
\begin{frame}
\frametitle{}
\begin{center}
\Huge GRACIAS
\end{center}
\end{frame}

\end{document}
