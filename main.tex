\documentclass[aspectratio=169]{beamer}
%Information to be included in the title page:
\title{Sistema de Gestión de Solicitudes Mesa de Ayuda "Helpdesk".}
\subtitle{Tesina para obtener el grado de Ingeniería en Tecnologías de la Información.}
\author{Juan Daniel Guerrero Guerrero}
\institute{Universidad Politécnica de Victoria}
\date{8 de Diciembre de 2025}

\usepackage{graphicx}
\usepackage{tikz}
\setbeamertemplate{sidebar right}{}
\setbeamertemplate{footline}{%
\hfill\usebeamertemplate***{navigation symbols}
\hspace{1cm}\insertframenumber{}/\inserttotalframenumber}

\usepackage[backend=biber, style=ieee, bibstyle=numeric, citestyle=numeric, sorting=none]{biblatex}
\addbibresource{references.bib}
\setbeamertemplate{bibliography item}{\insertbiblabel}


\begin{document}

\usebackgroundtemplate{%
%%\tikz\node[opacity=0.3] {\includegraphics[height=\paperheight,width=\paperwidth]{Diapositiva1}};}
\tikz\node {\includegraphics[height=\paperheight,width=\paperwidth]{Diapositiva1}};}
\thispagestyle{empty}
\frame{\titlepage}

\usebackgroundtemplate{%
%%\tikz\node[opacity=0.3] {\includegraphics[height=\paperheight,width=\paperwidth]{Diapositiva1}};}
\tikz\node {\includegraphics[height=\paperheight,width=\paperwidth]{Diapositiva2}};}

\begin{frame}
\frametitle{Contenido}
\tableofcontents
\end{frame}


\section{Introducción}
%%-------------------------------------------------------
\begin{frame}
\frametitle{Situación Actual en la DIT}
\begin{itemize}
\item Actualmente, el proceso de atención en la DIT carece de un sistema organizado.

\item Se utilizan canales informales (WhatsApp, verbal), lo que causa pérdida de información.

\item Existe duplicidad de tareas y falta de indicadores para la toma de decisiones.
\end{itemize}
\end{frame}
%%-------------------------------------------------------

%%-------------------------------------------------------
\begin{frame}
\frametitle{Objetivos}
\textbf{Objetivo General}
\begin{itemize}
\item[] Desarrollar un sistema web que permita gestionar de manera eficiente, centralizada y trazable las solicitudes.
\end{itemize}

\vspace{0.3cm}

\textbf{Objetivos Específicos}
\begin{itemize}
\item Implementar módulos de autenticación y asignación.
\item Generar reportes y visualización de indicadores (KPIs).
\end{itemize}
\end{frame}
%%-------------------------------------------------------

%%-------------------------------------------------------
\begin{frame}
\frametitle{Arquitectura del Sistema (MVC)}

Se utiliza una arquitectura Cliente-Servidor sobre el framework CakePHP.

\begin{itemize}
\item \textbf{Modelo:} Gestiona la lógica de datos y conexión a MySQL.

\item \textbf{Vista:} Plantillas que generan la interfaz de usuario.

\item \textbf{Controlador:} Intermediario que procesa las peticiones HTTP.
\end{itemize}

\vspace{0.5cm}

\begin{center}
% Ajusta el width (0.7) según el tamaño real de tu imagen
\includegraphics[width=0.4\textwidth]{diagrama_mvc_uml.png}
\end{center}

\end{frame}
%%-------------------------------------------------------



\section{Marco teórico}
%%-------------------------------------------------------
\begin{frame}
\frametitle{Actores y Funcionalidades}

El sistema interactúa con 4 actores principales: \textbf{Administrador}, \textbf{Jefe de Área}, \textbf{Agente Técnico} y \textbf{Solicitante}.

\vspace{0.3cm}

\textbf{Funcionalidades clave:}
\begin{itemize}
\item Registrar solicitud.
\item Asignar ticket.
\item Validar solución.
\item Visualizar Dashboard.
\end{itemize}

\vspace{0.2cm}

\begin{center}
\includegraphics[width=0.2\textwidth]{diagrama_caso_de_uso.png}
\end{center}

\end{frame}
%%-------------------------------------------------------


\section{Sistema Propuesto}
%%-------------------------------------------------------
\begin{frame}
\frametitle{Autenticación y Control de Acceso}

\begin{itemize}
\item Implementación de autenticación segura y control de acceso basado en roles (RBAC).

\item Validación de identidad mediante usuario y contraseña encriptada.
\end{itemize}

\vspace{0.5cm}

\begin{center}
\includegraphics[width=0.5\textwidth]{Login.png}
\end{center}

\end{frame}
%%-------------------------------------------------------

%%-------------------------------------------------------
\begin{frame}
\frametitle{Registro de Solicitudes}

\begin{itemize}
\item Formulario accesible para usuarios internos y externos.

\item Captura de datos críticos: Descripción del problema, tipo de solicitud y evidencia adjunta.

\item Validación automática de archivos adjuntos.
\end{itemize}

\vspace{0.5cm}

\begin{center}
\includegraphics[width=0.5\textwidth]{Formulario.png}
\end{center}

\end{frame}
%%-------------------------------------------------------

%%-------------------------------------------------------
\begin{frame}
\frametitle{Automatización y Notificaciones}

\begin{itemize}
\item Envío automático de correos electrónicos al crear un ticket o cambiar su estatus.

\item Incluye un enlace directo (token) para seguimiento sin necesidad de login para externos.
\end{itemize}

\vspace{0.5cm}

\begin{center}
\includegraphics[width=0.75\textwidth]{Notificación.png}
\end{center}

\end{frame}
%%-------------------------------------------------------




\section{Implementación y resultados}
%%-------------------------------------------------------
\begin{frame}
\frametitle{Panel de Control (Dashboard)}

\begin{itemize}
\item Visualización de Indicadores Clave de Desempeño (KPIs) en tiempo real.

\item Contadores de tickets asignados, nuevos y pendientes de aprobación.

\item Acceso rápido a reportes por área o agente.
\end{itemize}

\vspace{0.5cm}

\begin{center}
\includegraphics[width=0.5\textwidth]{Dashboard.png}
\end{center}

\end{frame}
%%-------------------------------------------------------

%%-------------------------------------------------------
\begin{frame}{Gestión y Asignación de Tickets}
    \begin{itemize}
        \item Listado general de tickets recibidos en el área de ``Recepción''.
        \item \textbf{Flujo de trabajo:} El jefe de área reasigna el ticket al departamento correcto o a un técnico específico.
        \item Visualización de estatus: \textit{Nuevo, En Proceso, Finalizado}.
    \end{itemize}

    \vspace{0.5cm}

    \begin{figure}
        \centering
        % Ajusta el ancho (width) según el espacio disponible en tu plantilla
        \includegraphics[width=0.5\textwidth]{Generación_y_visualización.png}
        \caption{Listado tabular de tickets}
    \end{figure}
\end{frame}
%%-------------------------------------------------------

%%-------------------------------------------------------
\begin{frame}{Seguimiento y Bitácora}
    \begin{itemize}
        \item Historial detallado de cada incidencia (\textbf{Bitácora de seguimiento}).
        \item Interacción tipo chat entre el agente técnico y el solicitante.
        \item El sistema impide cerrar un ticket sin la validación del jefe de área (\textit{Status: Pendiente de Aprobación}).
    \end{itemize}

    \vspace{0.5cm}

    \begin{figure}
        \centering
        \includegraphics[width=0.5\textwidth]{vista_seguimiento.png}
        \caption{Chat de interacción y detalles del ticket}
    \end{figure}
\end{frame}
%%-------------------------------------------------------

%%-------------------------------------------------------
\begin{frame}{Módulo Administrativo}
    \begin{itemize}
        \item Gestión de catálogos del sistema: Usuarios, Áreas, Tipos de Solicitud y Estatus.
        \item Panel exclusivo para usuarios con rol de \textbf{Administrador}.
        \item Permite la configuración global del sistema sin necesidad de modificar el código.
    \end{itemize}

    \vspace{0.5cm}

    \begin{figure}
        \centering
        \includegraphics[width=0.5\textwidth]{Vista_admin.png}
        \caption{Tablas de configuración y gestión de catálogos}
    \end{figure}
\end{frame}
%%-------------------------------------------------------

%%-------------------------------------------------------
\begin{frame}{Vista del Técnico}
    \begin{itemize}
        \item Bandeja personalizada para el agente técnico.
        \item Solo muestra los tickets asignados a su ID de empleado para focalizar el trabajo.
        \item Permite acciones rápidas: \textbf{Ver detalles} o \textbf{Cambiar estatus}.
    \end{itemize}

    \vspace{0.5cm}

    \begin{figure}
        \centering
        \includegraphics[width=0.5\textwidth]{gestion_tecnico.png}
        \caption{Vista filtrada para el agente}
    \end{figure}
\end{frame}
%%-------------------------------------------------------

\section{Conclusiones y trabajo futuro}

%%-------------------------------------------------------
\begin{frame}{Conclusiones}
    \begin{itemize}
        \item Se logró centralizar la información y eliminar la pérdida de datos.
        \item Se implementó exitosamente la trazabilidad y auditoría de acciones.
        \item El sistema provee métricas reales para evaluar la carga laboral y eficiencia de la DIT.
    \end{itemize}
    
    \vspace{1cm}
    
    % Espacio reservado para logo o cierre textual
    \begin{center}
        \textbf{Universidad Politécnica de Victoria}
    \end{center}
\end{frame}
%%-------------------------------------------------------

%%-------------------------------------------------------
\begin{frame}{Trabajo Futuro}
    \begin{itemize}
        \item \textbf{Módulo de Inventarios:} Desarrollar una vinculación directa entre los tickets de soporte y los números de serie del hardware para identificar equipos recurrentemente problemáticos.
        \item \textbf{Notificaciones en Tiempo Real:} Implementar tecnología WebSockets para que las alertas aparezcan inmediatamente en pantalla sin necesidad de recargar la página o revisar el correo.
        \item \textbf{Aplicación Móvil:} Crear una App nativa (o API REST) para que los técnicos de campo puedan actualizar el estatus y cerrar tickets desde su celular en el lugar de los hechos.
    \end{itemize}
\end{frame}
%%-------------------------------------------------------


\section*{References}
\begin{frame}[t,noframenumbering,plain,allowframebreaks]{Referencias}
    %\frametitle{Referencias}
    \AtNextBibliography{\tiny}
    \printbibliography
\end{frame}

% En la ultima diapositiva ponen de fondo la version de la portada
\usebackgroundtemplate{%
%%\tikz\node[opacity=0.3] {\includegraphics[height=\paperheight,width=\paperwidth]{Diapositiva1}};}
\tikz\node {\includegraphics[height=\paperheight,width=\paperwidth]{Diapositiva1}};}
\begin{frame}
\frametitle{}
\begin{center}
\Huge GRACIAS
\end{center}
\end{frame}

\end{document}
